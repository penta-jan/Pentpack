\documentclass[a5paper]{scrartcl}

\usepackage[blackness=10]{pentpack}


\title{Pentpack.sty}
\subtitle{A package to draw Pentagame board and moves}

\author{Jan `Penta' Suchanek}

\begin{document}

\maketitle
\begin{center}
        
    \begin{pentdiag}{0.4\linewidth}
        \foreach \s in {0,...,4}
        {
            \block{\s}{};
            \linie{\s}{\s a}{->};
            \linie{\s}{\fpeval{\s+1}}{-};
            \cbog{\s};
        }
    \end{pentdiag}

\end{center}
\begin{figure}
    \centering

    \begin{pentdiag}{6cm}
        \block{0}{black};
        \block{2}{WildStrawberry};
        \block{3}{Goldenrod};
        \linie{1}{3a}{->};
        \linie{2-3}{1a}{color=red};
        \linie{2a}{1a}{color=red,<-};
        \block{0a}{Goldenrod};
        \bogen{1}{};
        \cjunct{0};
        \cjunct{1};
        \bogen{3}{<->};
        \ccornr{2};
        \ccornl{0};
        \cbog{5};
        \block{3+2}{red};
        \block{2-3}{red};
        \block{2a1}{red};
        \block{2r2}{red};
    \end{pentdiag}

    \caption{Not so easy}
    \label{fig:mydiag}
\end{figure}

\subsection*{Loading the package}

This is a package to draw Pentagame boards. The design of the board can, to some extent, be manipulated upon loading the package; later, only the size of the individual board can be changed. 

    \verb|sw|  have a black-and-white board
    
    \verb|rotate=0| rotates the board with \verb|rotate=90| set as default
    
    \verb|linecolor=color| set the default color for all lines that are drawn; the default is black. 
    
    \verb|blackness=90| sets the shade of the board background, the default is 66.

\subsection*{Environment pentdiag}

\verb|\begin{pentdiag}{circlediameter}| etc. will draw you a Pentagame board in the size specified. Note that the parameter defines the circle, not the circumference, which is slightly larger; so setting it to \verb|\linewidth| will result in an overfull \verb|hbox|.

\subsection*{Nodes}

The corners are labelled counter-clockwise \verb|0| to \verb|5|, where both \verb|0| and \verb|5| is the white corner. This holds throughout. 

The junctions are labelled \verb|0a| to \verb|5a| respectively.

The environment creates a TikZ picture, so you can use all TikZ commands here. 

\subsection*{Lines}

\verb|\linie{from}{to}{options}| draws a line between two nodes or coordinates (the centre of the board is \verb|0,0|). Options can be of type \verb|->| or \verb|color=red| etc. can be passed to TikZ to make these arrows, coloured, dashed etc. 

\noindent\verb|\bogen{number}{options}| draws one of the arcs, where arc number \verb|1| is left of node number \verb|1| etc.; again, options can be passed. 

\subsection*{Stops}

The stops on the lines are labelled as follows:

    \verb|0r1| is the first stop on the ring from node \verb|0|, counter-clockwise.
    
    \verb|0+1| is the first stop on the long line from node \verb|0|, clockwise.
    
    \verb|0-1| is the first stop on the long line from node \verb|0|, counter-clockwise.
    
    \verb|0a1| is the first stop on the short line from node \verb|0a|, counter-clockwise.

\subsection*{Blocks}

\verb|\block{node}{color}| places a filled circle of size \verb|\svar| onto the stop or node specified. Default colour is \verb|black|.

Since you sometimes just want to block a line without bothering too much about the exact placement of the block, there are four `shortcut' commands:

\verb|\cbog{number}| blocks arc \verb|number|.

\verb|\ccornl{number}| blocks long line left from node  \verb|number|.

\verb|\ccornr{number}| blocks long line right from node  \verb|number|.
|

\verb|\cjunct{}| blocks short line \emph{opposite} node  \verb|number|.

\section*{Agenda}
\begin{itemize}
    \item migrate to \verb|arrows.meta| and have arrow tips scale, too!
\end{itemize}
\end{document}
