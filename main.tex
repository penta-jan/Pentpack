\documentclass[a5paper]{scrartcl}

\usepackage[brettrad=0.2\linewidth]{pentpack}

\usepackage{standalone}

\setlength{\parskip}{1ex}
\setlength{\parindent}{0pt}

\title{pentpack.sty}
\subtitle{A package to draw Pentagame board and moves}

\author{Jan `Penta' Suchanek}

\begin{document}

\maketitle
\begin{center}
        
    \begin{pentdiag}
        \foreach \s in {0,...,4}
        {
            \pblock{\s};
            \plinie{\s}{\s a}{->};
            \plinie{\s}{\fpeval{\s+1}}{-};
            \pcbog{\s};
        }
    \end{pentdiag}

\end{center}

\boardwhite

\begin{figure}
    \setboolean{details}{true}
    \centering

    \begin{pentdiag}[0.2\linewidth]
        \pblock[\myred]{2};
        \plinie{1}{3a}{->};
        \plinie{2-3}{1a}{color=red} node[midway,above] {1};
        \plinie{2a}{1a}{color=red,<-};
        \pblock[\myyellow]{4r1};
        \pbogen{1}{};
        \pcjunct{0};
        \pcjunct{1};
        \pbogen{3}{<->};
        \pccornr{2};
        \pccornl{0};
        \pcbog{5};
        \addplayers[<-,thick]{2};
        \draw[color=green] (p1) -- (p2);
    \end{pentdiag}

    \caption{Not so easy}
    \label{fig:mydiag}
\end{figure}

\boardcoloured

\begin{figure}[ht]
    \centering
    \documentclass{standalone}

\usepackage[]{pentpack}

\begin{document}


    \begin{pentdiag}
        \ptwo{0}{white}{white};
        \ptwo{1}{\mygreen}{\mygreen};
        \ptwo{2}{\myyellow}{\myyellow};
        \ptwo{3}{\myred}{\myred};
        \ptwo{4}{\myblue}{\myblue};
        \pblock{0a};
        \pblock{1a};
        \pblock{2a};
        \pblock{3a};
        \pblock{4a};
    \end{pentdiag}

\end{document}
    \caption{Initial position for two players}
    \label{fig:initialposition}
\end{figure}

\subsection*{Loading the package}

This is a package to draw Pentagame boards using TikZ. The design of the board can, to some extent, be manipulated upon loading the package; later, only the size of the individual board can be changed. 

\verb|sw|  have grey shades instead of colours

\verb|white|  make all nodes white 

\verb|noring|  omits the ring / border of the board

\verb|simple| omits individual stops on the lines and the circle and draws dotted and dashed lines instead. This happens by setting the boolean variable \verb|details| to false. You can use this switch (outside the pentdiag environment!) if you need an exception. If you do this within the \verb|figure| environment, the change remains local.
    
\verb|rotate=0| rotates the board with \verb|rotate=90| set as default; \verb|rotate=0| sets white to the right.
    
\verb|linecolor=color| set the default color for all lines that are drawn; the default is black. 
    
\verb|blackness=90| sets the shade of the board background, the default is 40.

Because it's TikZ, you can now draw all sort of things on it, ideally relative to the nodes, which are labeled as listed below.

\subsection*{Colours}

Colours are named \verb|\mywhite|, \verb|\mygreen|, \verb|\myyellow|, \verb|\myred| and \verb|\myblue|. In colour mode, these are loaded with \verb|[dvipsnames]{xcolor}|: white, SpringGreen, Goldenrod, WildStrawberry and Cerulean. These commands are set as options to TikZ as in \verb|[color=\mywhite]|. You can define your own by redefining these commands:

\verb|\renewcommand{\mywhite}{color}|

Ideally this would be done in your preamble.


\subsection*{Environment pentdiag}

\verb|\begin{pentdiag}{circlediameter}| etc.~draws a Pentagame board in the size specified which is \verb|\brettrad|. Note that the parameter defines the circle, not the circumference, which is slightly larger; so setting it to \verb|\linewidth| will result in an overfull \verb|hbox|.

\subsection*{Nodes}

The corners are labelled counter-clockwise \verb|0| to \verb|5|, where both \verb|0| and \verb|5| is the white corner.  

The junctions are labelled \verb|0a| to \verb|5a| respectively, also counter-clockwise.

The environment creates a TikZ picture, so you can use all TikZ commands here. All new commands defined by the package lack the closing \verb|;| so they can be complemented and need to be closed by a semicolon. 

The radius of the corners is of length \verb|\cvar| and the radius of the junctions \verb|\jvar|

\subsection*{Lines}

\verb|\plinie{from}{to}{options}| draws a line between two nodes or coordinates (the centre of the board is \verb|0,0|). Options can be of type \verb|->| or \verb|color=red| etc. can be passed to TikZ to make these arrows, coloured, dashed etc. They can be complemented for labeling, e.g.

\verb|\plinie{1}{3a}{->} node[midway,right] {1};|

---draws the number 1 next to the line.

\verb|\pbogen{number}{options}| draws one of the arcs, where arc number \verb|0| is left of node number \verb|0| etc.; again, options for colour and arrows can be passed. 

If you need an arc spanning just a part of an arc, here's the more precise command:

\verb|\pkbogen{number}{start}{stop}{options}| draws on the arc segment \verb|number| (as above) but from stop \verb|start| to stop  \verb|stop|. The arc thus drawn can in fact be longer than just one fifth of the circle. 

\subsection*{Stops}

The stops on the lines and arcs are labelled as follows:

\verb|0r1| is the first stop on the ring from node \verb|0|, counter-clockwise.
    
\verb|0+1| is the first stop on the long line from node \verb|0|, clockwise.
    
\verb|0-1| is the first stop on the long line from node \verb|0|, counter-clockwise.
    
\verb|0a1| is the first stop on the short line from node \verb|0a|, counter-clockwise.

The radius of the stops is of length \verb|\svar|

\subsection*{Blocks}

\verb|\pblock[colour]{node}| places a filled circle of size \verb|\svar| onto the stop or node specified. Default colour is \verb|black|.

Since you sometimes just want to block a line without bothering too much about the exact placement of the block, there are four `shortcut' commands:

\verb|\pcbog{number}| blocks arc \verb|number|.

\verb|\pccornl{number}| blocks long line left from node  \verb|number|.

\verb|\pccornr{number}| blocks long line right from node  \verb|number|.

\verb|\pcjunct{}| blocks short line \emph{opposite} node  \verb|number|.

The radius of the blocks is of length \verb|\fvar|.

\subsection*{Pieces}

To draw a triangle, use

\verb|\ptri{node}{colourtri}| for triangles. The colour should be of type \verb|\myred| etc.

\verb|\psqu{node}{coloursqu}| is the same for a square. 

To draw two pieces at a corner, use

\verb|\ptwo{corner}{colortri}{colorsqu}| which draws a triangle and a square in the two colours.



\section*{Highlighting}

\verb|\phighlight{node}| draws a circle of size \verb|3\svar| around a node and an arrow of length \verb|5\svar| pointing at the node; in simple mode the circle is suppressed. It seems that the arrows depending on the \verb|calc| package do not always work. But remember that you can virtually draw anything once you address the right node.

\section*{Players}

\verb|\addplayers[lineoptions]{number}| adds players written in \verb|\mathcal{A}| format in a counter-clockwise manner; this can be a number between two and five. Set \verb|lineoptions| for arrows between the players, e.g.~\verb|\addplayers[<-,ultra thick]{3}| for counter-clockwise arrows.

\section*{A blue print}

Use this to draw a complete two-player position:
\begin{verbatim}
    \begin{pentdiag}
        %\two{}{}{\mywhite}{\mywhite}
        \ptri{}{\mywhite};
        \ptri{}{\mygreen};
        \ptri{}{\myyellow};
        \ptri{}{\myred};
        \ptri{}{\myblue};
        \psqu{}{\mywhite};
        \psqu{}{\mygreen};
        \psqu{}{\myyellow};
        \psqu{}{\myred};
        \psqu{}{\myblue};
        \pblock{};
        \pblock{};
        \pblock{};
        \pblock{};
        \pblock{};
        \pblock[gray]{};
        \pblock[gray]{};
        \pblock[gray]{};
        \pblock[gray]{};
        \pblock[gray]{};
    \end{pentdiag}
\end{verbatim}

\section*{Agenda}
\begin{itemize}
    \item It would be better if the syntax of the pieces would include the colours in an optional, rather than a second mandatory parameter. This is something for later because it would break compatibility.
    \item Label all nodes in a redundant double manner as introduced in the book.
    \item I don't like the syntax for the nodes on the pentagon, it's too confusing. Sometimes things are labelled clockwise, sometimes counter-clockwise, which is no good. 
    \item Sometimes stuff starts at 0 and sometimes at 1, which is not consistent enough.
    \item Migrate to \verb|arrows.meta| and have arrow tips scale too.
\end{itemize}

\begin{figure}
    \centering
    \documentclass{standalone}

\usepackage[]{pentpack}

\begin{document}
    \begin{pentdiag}{7cm}%
        \foreach \s in {0,...,4}
        {
         %   \block{\s}{};
%            \linie{\s}{\s a}{->};
%            \linie{\s}{\fpeval{\s+1}}{-};
        %    \cbog{\s};
        }
        \block{4r2}{};
        \block{1-3}{};
        \block{1+4}{};
        \block{1a1}{};
        \kbogen{1}{2}{3}{->};
        \highlight{2a1};
        \highlight{3+2};
        \highlight{2r1};
        \tri{0}{white};
        \two{4}{\myred}{\myblue};
    \end{pentdiag}

\end{document}
    \caption{Further examples}
    \label{fig:furtherexamples}
\end{figure}

\end{document}
