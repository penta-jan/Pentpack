\documentclass[a5paper]{scrartcl}

\usepackage[rotate=0,blackness=70]{pentpack}


\title{Pentpack - Draw Basic Pentagame Moves}

\author{Jan `Penta' Suchanek}

\begin{document}

\maketitle
\begin{center}
        
    \begin{pentdiag}{0.618\paperwidth}
        \foreach \s in {0,...,4}
        {
            \block{\s}{};
            \linie{\s}{\s a}{->};
            %\bogen{\s}{<->};
        }
    \end{pentdiag}

\end{center}
\begin{figure}
    \centering

    \begin{pentdiag}{6cm}
        \block{0}{black};
        \block{2}{WildStrawberry};
        \block{3}{Goldenrod};
        \linie{1}{3a}{->};
        \linie{0a}{1a}{color=red};
        \linie{2a}{1a}{color=red,<-};
        \block{0a}{Goldenrod};
        \bogen{1}{};
        \cjunct{0};
        \cjunct{1};
        \bogen{3}{<->};
        \ccornr{2};
        \ccornl{2};
        \cbog{1};
        \cbog{3};
    \end{pentdiag}

    \caption{Not so easy}
    \label{fig:mydiag}
\end{figure}
The package can be loaded with the option \verb|sw| to have a black-and-white board. However, you can still add colour in your commands.

Use the environment \verb|\begin{pentdiag}{circlediameter}| which will draw you a Pentagame board. Note that the parameter changes the circle, not the circumference which is slightly larger.

Nodes are labelled 0 to 5 with 0=5 white. For lines, use

\verb|\linie{from}{to}{options}| 

where options can be of type \verb|->| or \verb|color=red| etc. passed to TikZ. 

Use 

\verb|\bogen{number}{options}| 

to draw one of the arcs, where arc number 1 is left of node number 1 etc. 

Blocks cannot be placed (yet) onto individual stops, but only on lines to be blocked. There are these commands:

\verb|\block{node}{options}| puts a block (or piece, if you colour it) onto a node. 

\verb|\ccornr{node}{options}| and \verb|\ccornl{node}{options}| closes the long line from a node on stop three to the right or left, respectively.

\verb|\cbog{number}| closes an arc; and 

\verb|\cjunct{number}| closes the short line opposite of node \verb|number|.

ToDo: more precise calculations, the \verb|calc| package sucks.

\end{document}
